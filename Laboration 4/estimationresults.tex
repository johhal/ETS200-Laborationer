Below follows the results of the failure estimation as well as the estimation method used. 
The following two variables will be estimated:
\begin{itemize}
 \item $N_{future}$ - Is the amount of errors in the future, i.e. in additionally 50\% time.
 \item $N_{tot}$ - Is the total amount of errors in the analysed system.
\end{itemize}

\subsection{TROPICO R-1500}
TROPICO R-1500 is a Brazilian Electronic Switching System with 1500 subscribers and in the file, \code{godata1.txt}, the failure count of said system is stored and will be analysed. A compilation of the results are found in Table~\ref{goelokumototable} in appendix.
\subsubsection*{Subjective Estimation} 
First the cumulative sum of the faults found in week 1 to week 50 were plotted, see Figure~\ref{cumulativegodata1} in appendix, to see if a failure over time trend could be found.
The inspectors identified an exponentially declining trend and estimated the two variables $N_{future}$ and $N_{tot}$ to be 500 and 600 faults respectively, see Table~\ref{goelokumototable}. 
\subsubsection*{Goel-Okumoto method}
To use the Goel-Okumoto method, described in section~\ref{methods}, a program, \code{go.m}, were used to calculate the variables $a=547.5$ and $b=0.0242$.
When estimating the $N_{future}$ variable the found values of $a$ and $b$ were inserted in the Goele-Okumoto formula and $t$ were increased to 75, since $N_{future}$ is the cumulative sum of the errors found in additionally 50\% time. 
$N_{tot} = a$ since it is the sum of all errors found in an infinite amount of time.
The estimation of $N_{future}$ and $N_{tot}$ was 458.1 and 547.5 respectively, see Table~\ref{goelokumototable} in appendix.
\subsubsection*{Real Data}

\label{realdatago}
To find the accuracy of the subjective estimation and the Goel-Okumoto method the real values of $N_{future}$ and $N_{tot}$ were found by analysing a file, \code{godata2.txt}, containing the failure count of 31 more weeks. 
The real value of $N_{future}$ is the sum of all errors up to weak 75, which gives $N_{future}=459$, which is very close to the estimated value of 458.1. $N_{tot}$ on the other hand cannot be verified since there is no, nor will it ever be, a file containing the errors found until infinity. However, the last value of \code{godata2.txt} is 461 and is found at week 81, compared to 468 at week 81 in the Goel-Okumoto estimation. Since this is within 1.5\% of the estimation, it will be assumed that Goel-Okumoto estimates the $N_{tot}$ good enough.

\subsection{Reliability Jelinski-Moranda}
In this exercise a file, \code{jmdata1.txt}, containing the failures found in a system as well as the time between them will be used.
A compilation of the results are found in Table~\ref{jelinskimorandatable}.

\subsubsection*{Subjective Estimation} 
First the cumulative sum of the time between failures and the number of failures were plotted, see Figure~\ref{cumulativejmdata1} in appendix. Here as well the inspectors identified an exponentially declining trend and subjectively estimated $N_{future}$ and $N_{tot}$ to be 230 and 500 respectively, see Table~\ref{jelinskimorandatable} in appendix.

\subsubsection*{Jelinski-Moranda Estimation}

To use the Jelinski-Moranda method, described in section~\ref{methods}, a program, \code{jm.m}, were used to calculate the variables $N=226.5$ and $\phi=1.18\times10^{-4}$. Since Jelinski-Moranda estimates the intensity of failures $N_{future}$ cannot be estimated using this method.
$N_{tot}$ on the other hand is the fault when the JM-method equals zero, because if the intensity of failures equals zero no new errors will occur, which gives $N_{tot}=226.5$.

\subsubsection*{Real Data}
\label{realdatajm}
To find the accuracy of the subjective estimation and the Jelinski-Moranda method the real values of $N_{future}$ and $N_{tot}$ were found by analysing a file, \code{jmdata2.txt}, containing an additional 57 failures as well as the time between them. Since the end time in \code{jmdata1.txt} was around 9000 time units the $N_{future}$ is the number of faults found after 13500 units of time. By plotting the cumulative sum of the time between errors in \code{jmdata2.txt} it was found that the real value of $N_{future}=200$. The real data value of $N_{tot}$ is $N_{tot}=207$ and occurs at time 16656 and with an failure intensity $\lambda=0.0025$. Even tough this value is a bit of from the estimated $N_{tot}$ it is assumed to be realistic since $\lambda=0.0025$ at this point and Figure~\ref{cumulativejmdata2} in appendix shows that the amount of failures has severely stagnated.