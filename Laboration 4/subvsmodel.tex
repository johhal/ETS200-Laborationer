By comparing the subjective estimates and reliability model estimates in Table~\ref{goelokumototable} and Table~\ref{jelinskimorandatable} against the real values in the respective tables, the accuracy of the respective models can be evaluated. 

For the Goel-Okomoto model, the subjective estimates for $N_{future}$ and $N_{tot}$ were found to be 500 and 600, respectively, as compared to the Goel-Okumoto estimates for $N_{future}$ and $N_{tot}$ of 458.1 and 547.5, respectively. Consequently, the Goel-Okumoto model turned out to give a very good approximation of the real values that were found to be 459 and 461 for $N_{future}$ and $N_{tot}$, respectively. 

As indicated in section~\ref{estimationresults}, the real value of $N_{tot}$ didn't exist and was therefore replaced by the value of week 81 from the real data, i.e, 461. Thus, the Goel-Okumoto was remarkably precise in its approximation of $N_{tot}=468$ at week 81. 

For the Jelinski-Moranda model, the subjective estimates for $N_{future}$ and $N_{tot}$ were found to be 500 and 600, respectively, as compared to the Goel-Okumoto estimates for $N_{future}$ and $N_{tot}$ of 458.1 and 547.5, respectively. Consequently, the Goel-Okumoto model turned out to give a very good approximation of the real values that were found to be 459 and 461 for $N_{future}$ and $N_{tot}$, respectively. 

