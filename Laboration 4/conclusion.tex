By comparing the subjective estimates and reliability model estimates in Table~\ref{goelokumototable} and Table~\ref{jelinskimorandatable} against the real values in the respective tables, the accuracy of the respective models can be evaluated. 

For the Goel-Okomoto model, the subjective estimates for $N_{future}$ and $N_{tot}$ were found to be 500 and 600, respectively, as compared to the Goel-Okumoto estimates for $N_{future}$ and $N_{tot}$ of 458.1 and 547.5, respectively. Consequently, the Goel-Okumoto model turned out to give a very good approximation of the real values that were found to be 459 and 461 for $N_{future}$ and $N_{tot}$, respectively. 

As indicated in section~\ref{realdatago}, the real value of $N_{tot}$ didn't exist and was therefore replaced by the value of week 81 from the real data, i.e, 461. Thus, the Goel-Okumoto was remarkably precise in its approximation of $N_{tot}=468$ at week 81. 

For the Jelinski-Moranda model, the subjective estimates for $N_{future}$ and $N_{tot}$ were found to be 230 and 500, respectively. As noted in section~\ref{jmestimation}, the value of $N_{future}$ could not be found due to the model only estimating failure intensity. Hence, only the estimate of $N_{tot}$ was found to be 226,5. When comparing these estimates against the real values of $N_{future}$ and $N_{tot}$, i.e, 200 and 207 respectively, the Jelinski-Moranda turned out to give a quite good approximation of $N_{tot}$, compared to the subjective estimate of the same value, while the subjective estimation of the $N_{future}$ actually was a pretty good guess. 

As a model for estimating the number of faults that can be found in the future, the Goel-Okumoto turned out to be very good. On the other hand, it is probably more interesting to estimate the failure intensity in a software development than the number of failures in the future or at infinity. 