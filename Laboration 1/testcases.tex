Five conditions were found to apply to the Triangle.java class and ten equivalence classes could then be derived from these as shown below:

\noindent Condition 1: The input must be a positive Integer.
\begin{itemize}
	\item EC1. Input is a positive Integer, valid.
	\item EC2. Input is a non-positive Integer, invalid.
\end{itemize}

\noindent Condition 2: The triangle must be equilateral.
\begin{itemize}
	\item EC3. All sides are equal, valid. 
	\item EC4. All sides are unequal,invalid. 
\end{itemize}

\noindent Condition 3: Two sides or more must be equal.  
\begin{itemize}
	\item EC5. Two sides or more are equal, valid. 
	\item EC6. No sides are equal, invalid. 
\end{itemize}

\noindent Condition 4: One angle must be 90 degrees. 
\begin{itemize}
	\item EC7. One angle is 90 degrees, valid. 
	\item EC8. No angle is 90 degrees, invalid. 
\end{itemize}

\noindent Condition 5: All sides are unequal. 
\begin{itemize}
	\item EC9. No sides are equal, valid. 
	\item EC10. Two or more sides are equal, invalid. 
\end{itemize}

The table~\ref{classestable} shows all conditions and their associated equivalence classes. No Boundary value analysis was performed due to the reasons described in section~ref{methods}. 


\begin{table}[!htb]
	\centering
	\label{classestable}
	\caption{Selected equivalence classes}
    \begin{tabular}{|l|l|l|}
        \hline
        Condition  & Valid equivalence classes & Invalid equivalence classes \\ \hline
        1          & EC1                       & EC2                         \\ 
        2          & EC3                       & EC4                         \\ 
        3          & EC5                       & EC6                         \\ 
        4          & EC7                       & EC8                         \\ 
        5          & EC9                       & EC10                        \\
        \hline
    \end{tabular}
\end{table}

\begin{table}[!htb]
	\label{testinputtable}
	\caption{Summary of test inputs using equivalence class partitioning}
    \begin{tabular}{|l|l|l|l|}
        \hline
        Test case identifier & Input values & Valid EC      & Invalid EC         \\ \hline
        1                    & 3,4,5        & EC1, EC7, EC9 & EC4, EC6           \\ 
        2                    & 0,0,0        & EC3, EC5      & EC2, EC8, EC10     \\ 
        3                    & 1,1,1        & EC1, EC3, EC5 & EC8, EC10          \\ 
        4                    & -1,-2,3      & EC9           & EC2, EC4, EC6, EC8 \\ 
        5                    & -1,-1,-1     & EC3, EC5      & EC2, EC8, EC10     \\ 
        6                    & 1,2,3        & EC1, EC9      & EC4, EC6, EC8      \\ 
        7                    & 0,1,1        & EC5           & EC2, EC8, EC10     \\ 
        8                    & -3,-4,-5     & EC9           & EC2, EC4, EC6, EC8 \\ 
        9                    & 2,4,4        & EC1, EC5      & EC8, EC10          \\ 
        10                   & 3,4,6        & EC1, EC9      & EC4, EC6, EC8      \\
        \hline
    \end{tabular}
\end{table}






