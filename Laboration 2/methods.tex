The goal of this laboration exercise was to test some common methods and techniques appropriate for Black-Box Testing.
According to ``Practical Software Testing'' there are five common methods:
\begin{itemize}
	\item Equivalence class partitioning
	\item Boundary value analysis
	\item State transition testing
	\item Cause and effect graphing
	\item Error guessing
\end{itemize}

\noindent However, during the laboration only Equivalence Class Partitioning and Boundary Value Analysis were tested and will therefore only be discussed further on.

During the laboration exercise a triangle class was tested. The fact that the input consisted of three Integer sides aggravated the testing and when using Equivalence Class Partitioning it was tricky to create appropriate partitions of the input.

The method isImpossible shall return true if it is impossible to build a triangle from the provided side lengths. Assuming a triangle is valid if all sides are greater than zero creates two partitions, i.e. sides are 0,0,0 and sides are 1,1,1. However, since the input can be three arbitrary Integers the cases when only one side is zero or less must be considered as an impossible triangle and this will create several more partitions. This made the Equivalence Class Partitioning method feel ungainly and it also didn't add any confidence in that the test cases produced were adequate.

The Boundary Value Analysis had similar limitations as Equivalence Class Partitioning when dealing with three arbitrary inputs. 
Using the same example as above, the Boundary Value Analysis would result in:
\begin{itemize}
	\item BLB = sides are 0,0,0
	\item LB = sides are 1,1,1
	\item ALB = sides are 2,2,2
\end{itemize}
There are no upper bounds in this example since a triangle's sides can be very long. 
Now, whilst LB is a valid triangle and BLB isn't, a triangle with sides 0,1,1 shouldn't be a valid triangle. But does the test cases formed from the boundary values ensure that the method will return true?

During this laboration exercise none of the methods used were successful in ensuring that the test cases were adequate.

However, if the input would have been only one Integer, or a String with a clearly stated interval, both of these techniques would have been very valuable to reduce the amount of effort needed to ensure that the test cases indeed were good enough.


Despite the fact that Equivalence Class Partitioning and Boundary Value Analysis were deemed inadequate for this exercise, they might still be better than the alternatives;
\begin{itemize}
\item Error guessing relies on the tester's experience and since it was close to zero, this method is inappropriate.
\item State transition testing could suffice to test the classify() method if the classification was the new state with the side lengths as input.
\item Cause-effect graph could be used in the same manner as the State transition testing, with the sides as the causes and the classification as the effect. However, this method seems a bit to ``heavy-weight'' for this simple class.
\end{itemize}