The main difference between white-box and black-box testing is the amount of knowledge of the structure of the system provided to the tester. In black-box testing the system is regarded, as the name suggest, as an black-box. This means that the tester can be provided by the following knowledge sources:
\begin{itemize}
	\item Requirements Document
	\item Different specifications
	\item Knowledge about the domain
	\item Defect Analysis
	\item different types of data
\end{itemize}

\noindent When white-box testing the tester will have a complete insight in the structure of the system, that is he will have access to the following knowledge sources:
\begin{itemize}
	\item High-level design
	\item Detailed design
	\item Control flow graphs
	\item The source code
\end{itemize}

\noindent There are pros and cons with both of these testing techniques since the amount of knowledge provided to the tester will affect him in various ways. When black-box testing the tester has little knowledge about how the system performs a task,  he only knows that the system shall perform a certain task. This gives him the chance to test the system as an ordinary user and make sure that all the functionality is fullfilled, thus confirming that the design of the system is adequate. 

However, only black-box testing a system is often insufficient since the tester is unable to perform any kind of unit testing nor find errors in the code.
White-box testing on the other hand is very good at finding code errors because the tester has access to the source code. However, since the tester has access to the code he can be influenced by it and design tests that fit the code rather than test the code.