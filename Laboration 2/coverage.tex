During the lab three different code coverage criteria were used to evaluate to which extent the code was tested; Statement coverage, Decision/branch coverage and Condition coverage. 

Statement coverage refers to how many of the statements of a tested unit that are executed at least once. Decision/branch coverage means how many of the branches of an unit, e.g. true or false, that are executed at least once by the tests. Here, a complete boolean expression inside an If-clause is considered one true-or-false statement. Finally, condition coverage tells if each one of the conditions in the code has been evaluated to both true and false. 

During the lab it turned out that the statement coverage criteria was insufficient as a metric of how well tested the code was. This, because statement coverage only evaluates one of the statements in an if-else-clause, and still reports 100 \% coverage. The decision/branch coverage gave a good view of how many of the different paths through the code that had been executed and thus directly illustrated where tests were missing. The condition coverage was less helpful in evaluating the code coverage, but pointed out one else-if-clause that was unreachable. 

Since the code in the lab was relatively simple the coverage criteria above were sufficient but for more complex programs a combination of for example condition and decision coverage criteria would be useful. 



