First of all, it is impossible to tell that you have found all defects in an inspected object. However, you can find quite enough defects. 
Therefore an accurate estimation of the total amount of defects will be how close the "good enough" limit it will come, and that can be hard to measure.

\subsection{Subjective method}
It is difficult to prove the accuracy of an subjective method, since it is basically a guess. However, an experienced tester could intuitively get a feeling of what should be "good enough", and possibly making this an accurate method. 
This will of course highly depend on the experience of the inspector as well as his knowledge in similar projects.

\subsection{Capture-recapture method}
It is to some extent possible to prove the accuracy of these method since they are based on mathematical models. These models can even take into account that different defects are harder to find as well as the different detection capability amongst the inspectors. The drawback of these models are, however, that they don't take into account the complexity of the project being inspected. That is if two inspectors found defects according to, lets say, matrix $M_{1}$ described in section~\ref{inspection} the capture-recapture method would give the same point estimation of the amount of defects whether it was a NexDate-program or an NASA-satelite.