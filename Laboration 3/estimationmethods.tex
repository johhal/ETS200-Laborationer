During the laboratory session two different groups of estimation methods were used; subjective and capture-recapture methods. 

\subsection{Subjective methods}
A subjective method means that an inspector makes a qualified guess on the total defects in the object. For example, if the inspector finds 25 errors in the Requirements Document during the course of one week, he may conclude that there is probably not more than 30 defects in total in the Requirements Document. This technique can be effective if the inspector is experienced and the object to be inspected is rather small and trivial.

\subsection{Capture-recapture methods}
The Capture-recapture methods take a more scientific approach to find an estimation of the total amount of defects in the inspected object. The basic idea is to let two or more inspectors inspect an object individually during a fixed amount of time and note the defects they find. Then, the inspectors compile their results and discuss them to rule out any false positives and create an \textit{InspectionData} matrix.

\subsubsection{M0-ML Maximum likelihood}
By using the M0 model together with the Maximum likelihood estimator the total amount of defects in an inspected object can be calculated. 
According to the MO model all faults have equal detection probability and all inspectors have equal detection ability.

\subsubsection{Mt-ML Maximum likelihood}
By using the Mt model together with the Maximum likelihood estimator the total amount of defects in an inspected object can be calculated. 
According to the Mt model all faults have equal detection probability but the inspectors may have different detection abilities.

\subsubsection{Mh-JK Jack-knife}
By using the Mh model together with the Jack-knife estimator the total amount of defects in an inspected object, the standard defect deviation as well as the confidence interval of the fault content can be calculated. 
According to the Mh model all faults may have different detection probabilities but all inspectors have equal detection ability.  
